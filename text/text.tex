\section{Introduction and statement of the
problem}\label{introduction-and-statement-of-the-problem}

\label{sec:intro}

Enunciado do problema: Qual será o problema tratado pelo projeto e qual
sua importância? Qual será a contribuição para a área se bem sucedido?
Cite trabalhos relevantes na área, conforme necessário.

\section{Expected {[}scientific{]}
results}\label{expected-scientific-results}

\label{sec:project}

Resultados esperados: O que será criado ou produzido como resultado o
projeto proposto? Como os resultados serão disseminados?

Through refereed publications

\section{Scientific challenges and how to overcome
them}\label{scientific-challenges-and-how-to-overcome-them}

\label{sec:results}

Desafios científicos e tecnológicos e os meios e métodos para
superá-los: explicite os desafios científicos e tecnológicos que o
projeto se propõe a superar para atingir os objetivos. Descreva com que
meios e métodos estes desafios poderão ser vencidos. Cite referências
que ajudem os assessores que analisarão a proposta a entenderem que os
desafios mencionados não foram ainda vencidos (ou ainda não foram
vencidos de forma adequada) e que poderão ser vencidos com os métodos e
meios da proposta em análise.

\section{Timeline and management
plan}\label{timeline-and-management-plan}

\label{sec:time}

We list below the detailed schedule of the PhD project, appropriate for
4 years. The plan involves an anticipated ``est'agio doutorado
sandu'iche'' at UC Berkeley or UMD as soon as possible, such that the
student obtains expertise in using HARM with one of the developers of
the code.

PUT NICE TABLE WITH CRONOGRAMA

\section{Feasibility {[}include?{]}}\label{feasibility-include}

\label{sec:feasible}

In this section, we discuss the feasibility of the project.

\section{Team {[}include?{]}}\label{team-include}
